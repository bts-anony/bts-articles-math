\documentclass[12pt]{ctexbook}

\usepackage{amsmath}
\usepackage{physics}
\usepackage{tikz}
\usepackage{graphics}
\usepackage{array}
\usepackage{xcolor}
\usepackage{amssymb}
\usepackage{listing}
\usepackage{amsfonts}
\usepackage{mathrsfs}
\usepackage[backend=bibtex]{biblatex}
\usepackage{xeCJK}

\newtheorem{definition}{definition}
\numberwithin{definition}{section}
\newtheorem{theorem}{theorem}
\numberwithin{theorem}{section}
\newtheorem{exercise}{exercise}
\numberwithin{exercise}{section}
\newtheorem{example}{example}
\numberwithin{example}{section}
\newtheorem{lemma}{lemma}
\numberwithin{lemma}{section}

\begin{document}
    \chapter{线性空间}

    \section{定义}

    \begin{definition}
        [线性空间] 假定有一个数域\(\mathbb{F}\)\footnotemark{}, 给出一个集合\(V\), 并定义两个运算:
        \begin{enumerate}
            \item \(\forall \boldsymbol{\alpha} , \boldsymbol{\beta} \in V \)有\( \boldsymbol{\alpha} + \boldsymbol{\beta} \in V\)
            \item \(\forall \lambda \in \mathbb{F}, \boldsymbol{\alpha} \in V \)有\( \lambda \boldsymbol{\alpha} \in V\)
        \end{enumerate}
        且满足如下8条性质:
        \begin{enumerate}
            \item \(\forall \boldsymbol{\alpha} , \boldsymbol{\beta} , \boldsymbol{\gamma} \in V \)有\( (\boldsymbol{\alpha} + \boldsymbol{\beta}) + \boldsymbol{\gamma} = \boldsymbol{\alpha} + (\boldsymbol{\beta} + \boldsymbol{\gamma})\)
            \item \(\forall \boldsymbol{\alpha} , \boldsymbol{\beta} \in V\)有\( \boldsymbol{\alpha} + \boldsymbol{\beta} = \boldsymbol{\beta} + \boldsymbol{\alpha}\)
            \item \(\forall \lambda \in \mathbb{F} , \boldsymbol{\alpha} , \boldsymbol{\beta} \in V\)有\(\lambda (\boldsymbol{\alpha}\boldsymbol{\beta}) = \lambda \boldsymbol{\alpha} + \lambda \boldsymbol{\beta}\)
            \item \(\forall \lambda , \mu \in \mathbb{F}, \boldsymbol{\alpha} \in V\)有\((\lambda + \mu)\boldsymbol{\alpha} = \lambda\boldsymbol{\alpha} + \mu\boldsymbol{\alpha}\)
            \item \(\forall \lambda , \mu \in \mathbb{F}, \boldsymbol{\alpha} \in V\)有\((\lambda\mu)\boldsymbol{\alpha} = \lambda(\mu\boldsymbol{\alpha})\)
            \item \(\exists \boldsymbol{0} \in V\)有\(\forall \boldsymbol{A} \in V\)有\(\boldsymbol{A} + \boldsymbol{0} = \boldsymbol{A}\)
            \item \(\forall \boldsymbol{\alpha} \in V\)有\(\exists (-\boldsymbol{\alpha})\)有\(\boldsymbol{\alpha} + (-\boldsymbol{\alpha}) = \boldsymbol{0}\)
            \item \(\forall \boldsymbol{\alpha} \in V\)有\(1\boldsymbol{\alpha} = \boldsymbol{\alpha}\)
        \end{enumerate}
        则称\(V\)是数域\(\mathbb{F}\)上的线性空间, 其中\(V\)中的元素称为向量.\footnotemark{}
    \end{definition}

    \footnotetext{数域\(\mathbb{F}\)是指满足加法交换律、乘法交换律、加法结合律、乘法结合律、分配律、存在零元、幺元、存在加法逆元、乘法逆元的集合. 常见的数域有实数域\(\mathbb{R}\)、复数域\(\mathbb{C}\)、有理数域\(\mathbb{Q}\)、模p域\(\mathbb{Z}_p\).}
    \footnotetext{我们常用加粗或箭头表示向量, 以后我们一律将加粗字母表示向量, 不加粗的字母表示数域中的元素.}

    线性空间十分常见, 比如复数\(\mathbb{C}\)就构成\(\mathbb{R}\)上的线性空间, 空间选定坐标原点后, 就是\(\mathbb{R}\)上的线性空间. 有了线性空间, 我们要尝试去表示线性空间
    中的元素, 我们根据线性空间的性质, 充分利用加法与数乘, 将向量展开成线性组合, 将向量表示成如下形式:

    \begin{equation}
        \boldsymbol{\alpha} = \sum \lambda_i \boldsymbol{e}_i
    \end{equation}

    我们得到了\(\boldsymbol{\alpha}\)关于向量组\(\boldsymbol{e}_i\)的表示\(\lambda_i\),
    在确定了向量组\(\boldsymbol{e}_i\)后, 有两件事情是值得考虑的:

    \begin{enumerate}
        \item 是否所有向量都可以用这个向量组表示?
        \item 是否同一向量的表示唯一?
    \end{enumerate}
    
    先考虑第一个问题, 我们有了一个向量组\(S = \{\boldsymbol{e}_i\}\), 假设存在向量\(\boldsymbol{\alpha}\)不能够被向量组\(S\)表示, 那么我们可以将\(\boldsymbol{\alpha}\)加入
    \(S\), 得到新的向量组\(S \bigcup \{\boldsymbol{\alpha}\}\).

    再考虑第二个问题, 假设存在一个向量\(\boldsymbol{\alpha}\)的表示不唯一, 就有:

    \begin{equation}
        \boldsymbol{\alpha} = \sum \lambda_i \boldsymbol{e}_i = \sum \mu_i \boldsymbol{e}_i
    \end{equation}
    且\(\lambda_i\)不全等于\(\mu_i\).

    移项做差, 则有不全为0的数\(\nu_i\), 满足
    \begin{equation}
        \sum \nu_i \boldsymbol{e}_i = \boldsymbol{0} 
    \end{equation}

    \begin{definition}
        [线性相 (无) 关] 假设一组向量\(\boldsymbol{e}_i\)满足存在\(\lambda_i\):
        \begin{equation}
            \sum \lambda_i \boldsymbol{e}_i = \boldsymbol{0}
        \end{equation}
        则称向量组\(\boldsymbol{e}_i\)线性相关, 否则称线性无关.
    \end{definition}

    那么就有线性无关的向量组一定能够唯一的表示向量, 再考虑第一个问题中将\(\boldsymbol{\alpha}\)加入向量组\(S\),
    由于不存在不全为0的\(\lambda_i\)与\(k \neq 0\), 满足:

    \begin{equation}
        k (\boldsymbol{\alpha} + \sum (-\frac{\lambda_i}{k}) \boldsymbol{e}_i) = \boldsymbol{0}
    \end{equation}

    任何一个线性空间都能找出一组线性无关向量组\(\boldsymbol{e}_i\)能表示该线性空间的所有向量, 
    (其底层证明太过复杂, 但是读者可以从上述窥见一二, 不做要求).

    \begin{definition}
        [线性空间的基] 假设一组向量\(\boldsymbol{e}_i\)线性无关, 且能够唯一的表示线性空间中的所有向量, 则称向量组\(\boldsymbol{e}_i\)为线性空间的基.
        基中的向量称为基向量.
    \end{definition}

    \begin{definition}
        [线性空间的维数] 假设线性空间的基为\(\boldsymbol{e}_i\), 则称基向量的个数 (有限) 为线性空间的维数, 用\(\dim {(U)}\)表示.
    \end{definition}

    \begin{theorem}
        [维数存在] 假定有两组基\(\boldsymbol{e}_i\)与\(\boldsymbol{\epsilon}_j\), 
        且他们元素个数不相同, 不妨设\(\boldsymbol{e}_i\)的元素个数为\(n\), \(\boldsymbol{\epsilon}_j\)的元素个数为\(m (> n)\)

        也就是证明, 任取\(n\)元向量组\(\boldsymbol{v}_i\), 和\(m (> n)\)元向量组\(\boldsymbol{\nu}_j\)
        满足\(\boldsymbol{\nu}_j = \sum \lambda_{ji}  \boldsymbol{v}_i\)
        则存在整数\(\phi_j\), 满足\(\sum \phi_j \boldsymbol{\nu}_j = \boldsymbol{0}\)

        采用数学归纳法, 当\(n = 0\)时结论显然.

        当\(n = k + 1\)时, 考虑到\(\boldsymbol{\nu}_{m} \neq \boldsymbol{0}\), 所以
        存在\(\lambda_{mi} \neq 0\) (不妨设为\(\boldsymbol{v}_n\)), 考察向量组\(\boldsymbol{v}_1, \boldsymbol{v}_2, \ldots , \boldsymbol{v}_{n-1}\)
        与\(\boldsymbol{\nu}_0 - \frac{\lambda_{0n}}{\lambda_{mn}} \boldsymbol{v}_n, \boldsymbol{\nu}_1 - \frac{\lambda_{1n}}{\lambda_{mn}} \boldsymbol{v}_n, \ldots, \boldsymbol{\nu}_{m-1}, - \frac{\lambda_{(m-1)n}}{\lambda_{mn}} \boldsymbol{v}_n\)
        
        由归纳假设, 存在\(\phi_j\), 满足\(\sum \phi_j(\boldsymbol{\nu}_i - \frac{\lambda_{in}}{\lambda_{mn}} \boldsymbol{v}_n) = \boldsymbol{0}\), 于是\(\sum \phi_j\boldsymbol{\nu}_i - (\sum \phi_j\frac{\lambda_{in}}{\lambda_{mn}}) \boldsymbol{v}_n = \boldsymbol{0}\)
    \end{theorem}

    至此选取一组基向量, 就能够唯一的表示线性空间中的所有向量, 于是我们可以如下表示空间中的一个向量.

    \begin{equation}
        \boldsymbol{v} = \begin{bmatrix}
            v_1 \\
            v_2 \\
            \vdots \\
            v_n
        \end{bmatrix}
    \end{equation}

    \section{子空间}

    \begin{definition}
        [子空间] \(V\)是数域\(\mathbb{F}\)上的线性空间, 假设\(V\)有一个子集\(U\)满足8条性质, 则称\(U\)为\(V\)的子空间.
    \end{definition}

    任何一个线性空间\(V\)都有两个平凡子空间, 即\(V\)和\(\{\boldsymbol{0}\}\)

    \begin{definition}
        [向量组所张成的子空间] 线性空间\(V\)上有一个向量组\(\boldsymbol{v}_i\), 
        那么取集合\(\{\boldsymbol{\alpha} | \forall \lambda_i \in \mathbb{F}, \boldsymbol{\alpha} = \sum \lambda_i \boldsymbol{v}_i\}\)
        显然构成一个字空间, 称为向量组\(\boldsymbol{v}_i\)所张成的子空间.
    \end{definition}

    \begin{definition}
        [商空间] 假设\(V\)是数域\(\mathbb{F}\)上的线性空间, \(U\)是\(V\)的子空间, 取\(\boldsymbol{\alpha} \in V\)考察如下一群集合
        \(\{\boldsymbol{\alpha} + \boldsymbol{\gamma} | \forall \boldsymbol{\gamma} \in U\}\)
        定义数乘和加法运算.
        \begin{equation}
            \lambda \{\boldsymbol{\alpha} + \boldsymbol{\gamma} | \forall \boldsymbol{\gamma} \in U\} = 
            \{\lambda \boldsymbol{\alpha} + \boldsymbol{\gamma} | \forall \boldsymbol{\gamma} \in U\}
        \end{equation}
        \begin{equation}
            \{\boldsymbol{\alpha} + \boldsymbol{\gamma} | \forall \boldsymbol{\gamma} \in U\} + \{\boldsymbol{\beta} + \boldsymbol{\gamma} | \forall \boldsymbol{\delta} \in U\} = 
            \{\boldsymbol{\alpha} + \boldsymbol{\beta} + \boldsymbol{\gamma} | \forall \boldsymbol{\gamma} \in U\}
        \end{equation}
        则称\(\{\boldsymbol{\alpha} + \boldsymbol{\gamma} | \forall \boldsymbol{\gamma} \in U\}\)为\(V\)关于\(U\)的商空间, 记为\(V/U\)
    \end{definition}

    \begin{theorem}
        [商空间的维度] 假设\(V\)是数域\(\mathbb{F}\)上的线性空间, \(U\)是\(V\)的子空间, 则\(V/U\)的维度为\(\dim(V/U) = \dim(V) - \dim(U)\)
    \end{theorem}

    证明: 取\(V/U\)与\(U\)上的一组基\(\boldsymbol{e}_i\), \(\boldsymbol{\epsilon}_j\), 
    对每个元素\(\boldsymbol{e}_i\)选取一个代表元\(\boldsymbol{\varepsilon}_i\), 构成一组线性无关向量组.

    如果向量组\(\boldsymbol{\varepsilon}_i,\boldsymbol{\epsilon}_j\)线性相关, 则存在\(\lambda_i, \nu_j \in \mathbb{F}\)使得
    \(\sum \lambda_i \boldsymbol{\varepsilon}_i = \nu_j \boldsymbol{\epsilon}_j\), 于是\(\sum \lambda_i \boldsymbol{e}_i = \boldsymbol{0}\), 矛盾.

    另一方面, \(V\)的每一个元素\(\boldsymbol{\alpha}\)都是一个\(\{\boldsymbol{\alpha} + \boldsymbol{\gamma} | \forall \boldsymbol{\gamma} \in U/V\}\)中的元素, 
    用\(\boldsymbol{e}_i\)表示, 得到\(\sum \lambda_i \boldsymbol{\varepsilon}_i \in \{\boldsymbol{\alpha} + \boldsymbol{\gamma} | \forall \boldsymbol{\gamma} \in U/V\}\),
    而\(\sum \lambda_i \boldsymbol{\varepsilon}_i - \boldsymbol{\alpha}\)显然是\(U\)的元素.

    所以, \(\boldsymbol{\varepsilon}_i, \boldsymbol{\epsilon}_j\)是原空间的一组基, 带入立证.

    \begin{example}
        [子空间的例子] 考虑空间中自由电子在均匀磁场的运动, 这个时候我们就可以把电子的运动看成是一个向量,
        把垂直于磁场方向的平面看成是一个子空间, 电子在这个平面上的运动就是这个子空间上的运动.其商空间就是一个个平面,
        电子均匀的通过一个个平面.
    \end{example}

    \section{同态与同构}

    \begin{definition}
        [线性空间的同态] 假设\(V\)和\(W\)是数域\(\mathbb{F}\)上的线性空间, \(f: V \rightarrow W\)是一个线性映射, 如果对于任意的\(\boldsymbol{v}_1, \boldsymbol{v}_2 \in V\)
        \begin{equation}
            f(\boldsymbol{v}_1 + \boldsymbol{v}_2) = f(\boldsymbol{v}_1) + f(\boldsymbol{v}_2)
        \end{equation}
        \begin{equation}
            f(\lambda \boldsymbol{v}) = \lambda f(\boldsymbol{v})
        \end{equation}
        则称\(f\)是线性空间\(V\)到\(W\)的同态.
    \end{definition}

    考虑分析问题常见的同态, 比如平面力学问题直接把三维空间拍扁成二维平面, 
    就是\(\mathbb{R}\)上的三维线性空间到二维线性空间的同态, 垂直平面的方向的信息被忽略.

    \begin{definition}
        [同态核] 假设\(V\)和\(W\)是数域\(\mathbb{F}\)上的线性空间, \(f: V \rightarrow W\)是同态的映射,
        则称\(\ker(f) = \{\boldsymbol{v} | f(\boldsymbol{v}) = \boldsymbol{0}\}\)是\(f\)的同态核, 也称为\(f\)的零空间.
    \end{definition}

    此处同态核就是垂直平面的那根轴线. 另一方面, 二维空间也可以反过来投射到三维空间, 但此时必然不是满射.

    \begin{definition}
        [同态像] 假设\(V\)和\(W\)是数域\(\mathbb{F}\)上的线性空间, \(f: V \rightarrow W\)是同态的映射,
        则称\(\Im (f) = \{f(\boldsymbol{v}) | \forall \boldsymbol{v} \in V f(\boldsymbol{v})\}\)是\(f\)的同态像.
    \end{definition}

    像, 就是全体元素一起投影的结果.

    \begin{definition}
        如果一个同态\(f\)是双射, 则称\(f\)是同构, 记作\(\cong\).
    \end{definition}

    同构, 就是几乎完全相同, 一个线性空间具有的性质另一个也有, 所以大多数问题我们只需要在同构意义下考虑,
    那么什么样的线性空间同构呢?

    \begin{theorem}
        [同构定理] 假设\(V\)和\(W\)是数域\(\mathbb{F}\)上的线性空间, 且维数相同, 
        任取两者的一组基, \(\boldsymbol{v}_i, \boldsymbol{w}_i\), 映射\(f\)把
        \(\sum \lambda_i \boldsymbol{v}_i\)映射到\(\sum \lambda_i \boldsymbol{w}_i\)
        容易证明, 这是一组同构.
        
        另一方面, 假设\(V\)和\(W\)同构, 则\(\Im (f)\)等于向量组
        \(f(\boldsymbol{v}_i)\)所张成的子空间, 显然有\(\dim (V) \geq \dim (W)\)
        同理\(\dim (W) \geq \dim (V)\)所以\(V\)和\(W\)维数相同.
    \end{theorem}

    所以维数相同的线性空间同构, 我们只需考虑一个线性空间的维数即可.

    \begin{theorem}
        [同态基本定理] \(U/\ker (f) \cong \Im (f)\)
    \end{theorem}

    同态基本定理意味着, 一个同态所保留的信息, 等同于其原有的信息减去起舍去的信息.

    \begin{definition}
        [内直和] 假设\(U_1\), \(U_2\)是线性空间\(V\)的子空间, 满足
        \begin{equation}
            U_1 \cap U_2 = \boldsymbol{0}
        \end{equation}
        考虑\(\{\boldsymbol{\alpha} | \forall \boldsymbol{v}_1 \in U_1, \boldsymbol{v}_2 \in U_2, \boldsymbol{\alpha} = \boldsymbol{v}_1 + \boldsymbol{v}_2\}\)
        显然也是\(V\)的子空间, 称作\(U_1\)与\(U_2\)的内直和, 记作\(U_1 \oplus U_2\).
    \end{definition}

    \begin{definition}
        [外直和] 假设\(U_1\), \(U_2\)是线性空间,
        考察\(U_1 \times U_2\)\footnotemark{}上的元素\((\boldsymbol{v}_1, \boldsymbol{v}_2)\),
        定义数乘和加法运算为对应分量分别数乘和加法, 显然也构成线性空间, 称为\(U_1\)与\(U_2\)的外直和, 记作\(U_1 \oplus U_2\). 
    \end{definition}
    \footnotetext{两个集合相乘, 也就是笛卡尔积, 定义为: \(A \times B = \{(a, b) | a \in A, b \in B\}\)}

    \begin{definition}
        [直积] 假设\(U_1\), \(U_2\)是线性空间,
        考察\(U_1 \times U_2\)上的元素\((\boldsymbol{v}_1, \boldsymbol{v}_2)\),
        定义数乘和加法运算为对应分量分别数乘和加法, 显然也构成线性空间, 称为\(U_1\)与\(U_2\)的直积, 记作\(U_1 \otimes U_2\). 
    \end{definition}

    大家可能会疑惑, 为什么外直和和直积的定义一样, 因为我们上述只考虑了两个线性空间
    外直和和直积的定义, 但是我们可以把直和和直积的定义推广到任意个线性空间, 在无限个线性空间时,
    情况就有所不同了, 外直和只允许有限个分量不为\(\boldsymbol{0}\), 而外直积允许无限个分量不为\(\boldsymbol{0}\).

    刚才提到的同态同构子空间, 基本上都是得到更小的空间, 而直和和直积都是得到更大的空间.

    \begin{theorem}
        [直和和直积的维数] 
        \begin{equation}
            \dim{(U_1 \oplus U_2)} = \dim{(U_1 \otimes U_2)} = \dim{U_1} + \dim{U_2}
        \end{equation}
    \end{theorem}

    另外, 显然有
    \begin{equation}
        U/V \otimes V \cong U
    \end{equation}
    \begin{equation}
        U/V \otimes V/W \cong U/W
    \end{equation}

    \begin{exercise}
        \begin{enumerate}
            \item 证明以上结论
            \item 求证\(\mathbb{C}\)上线性空间都是\(\mathbb{R}\)上的线性空间.
            \item 当\(\mathbb{R}\)与\(\mathbb{R}^2\)视作\(\mathbb{Q}\)上的线性空间时, 证明它们同构.
            \item 举出物理里面线性空间的例子.
        \end{enumerate}
    \end{exercise}
\end{document}
