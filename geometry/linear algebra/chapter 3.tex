\documentclass[12pt]{ctexbook}

\usepackage{amsmath}
\usepackage{physics}
\usepackage{tikz}
\usepackage{graphics}
\usepackage{array}
\usepackage{xcolor}
\usepackage{amssymb}
\usepackage{listing}
\usepackage{amsfonts}
\usepackage{mathrsfs}
\usepackage[backend=bibtex]{biblatex}
\usepackage{xeCJK}
\usepackage{arydshln}
\usepackage{tabularray}

\newtheorem{definition}{definition}
\numberwithin{definition}{section}
\newtheorem{theorem}{theorem}
\numberwithin{theorem}{section}
\newtheorem{exercise}{exercise}
\numberwithin{exercise}{section}
\newtheorem{example}{example}
\numberwithin{example}{section}
\newtheorem{lemma}{lemma}
\numberwithin{lemma}{section}

\begin{document}
    \chapter{矩阵}

    \section{定义}

    上一章我们已经提到了, 矩阵的概念.

    \begin{definition}
        [矩阵] 数域\(\mathbb{F}\)上的\((n,m)\)类矩阵是指\((n,m)\)个数域\(\mathbb{F}\)上的数, 排列成\(n\)行\(m\)列的整体.
        全体\((n,m)\)矩阵用\(\mathbb{F}^{n\times m}\)表示.
    \end{definition}

    我们引入矩阵, 使用的是线性空间的同态, 和``张量'', 那么自然就引入了一个线性空间的结构.

    \begin{definition}
        [矩阵加法] 对于矩阵\(\mathbf{A},\mathbf{B}\in\mathbb{F}^{n\times m}\), 定义\(\mathbf{A}+\mathbf{B}\)的对应元素为\(\mathbf{A},\mathbf{B}\)对应元素的和.
    \end{definition}

    \begin{definition}
        [矩阵数乘] 对于矩阵\(\mathbf{A}\in\mathbb{F}^{n\times m}\), 定义\(\alpha\mathbf{A}\)的对应元素为\(\alpha\mathbf{A}\)对应元素的数乘.
    \end{definition}

    我们用\(A_{ij}\)表示矩阵\(\boldsymbol{A}\)第\(i\)行第\(j\)列的元素.

    上述式子又可以写成:
    
    \begin{equation}
        {(A+B)}_{ij} = A_{ij} + B_{ij}
    \end{equation}
    \begin{equation}
        {(\alpha A)}_{ij} = \alpha A_{ij}
    \end{equation}

    当然, 矩阵同构于映射, 那么映射最重要的性质之一就是映射可以叠加.
    
    有三个线性空间\(\mathcal{U}\), \(\mathcal{V}\), \(\mathcal{W}\), 
    两个同态\(\mathbf{A}:\mathcal{U}\to\mathcal{V}\), \(\mathbf{B}:\mathcal{V}\to\mathcal{W}\),
    定义\(\mathbf{B}\mathbf{A}:\mathcal{U}\to\mathcal{W}\)为\(\mathbf{B}\mathbf{A}(x) = \mathbf{B}(\mathbf{A}(x))\).

    将上述过程按分量展开, 得到\({(BA)}_{ij} = \sum_{k} {B}_{ik} {A}_{kj}\)

    \begin{definition}
        [矩阵乘法] 设\(\mathbf{A} \in \mathbb{F}^{n \times m}\), \(\mathbf{B} \in \mathbb{F}^{m \times p}\), 则\(\mathbf{A}\mathbf{B} \in \mathbb{F}^{n \times p}\).
        \begin{equation}
            {(AB)}_{ij} = \sum_{k = 1}^{m} {A}_{ik} {B}_{kj}
        \end{equation}
    \end{definition}

    矩阵乘法如同映射的叠加, 也具有结合律.

    一类重要的同态是同构, 同构中也有自同构, 自同构可以多次施行.

    \begin{definition}
        [矩阵幂次] 设\(\mathbf{A} \in \mathbb{F}^{n \times n}\), 则\(\mathbf{A}^k \in \mathbb{F}^{n \times n}\).
        \begin{equation}
            \mathbf{A}^k = \mathbf{A} \mathbf{A} \cdots \mathbf{A}
        \end{equation}
    \end{definition}

    但是并不是所有\(n\)阶方阵都代表同构, 考察如下矩阵:

    \begin{equation}
        \begin{bmatrix}
            1 & 0 \\
            0 & 0
        \end{bmatrix}
    \end{equation}

    很显然, 这个矩阵所代表的线性映射并不是一种同构, 所以, 接下来我们要考虑什么样的
    方阵能代表同构.

    \section{行列式}

    假设有一个矩阵\(\mathbf{M}\)与其对应的线性映射\(f:\mathcal{U} \to \mathcal{V}\), 
    显然, 只要维数匹配上了就是同构, 如果维数匹配上了, 那么\(\Im \mathcal{U}\)自然而然就有维数最高的``体积''.

    \begin{definition}
        [行列式] \(f:\mathcal{U}\to\mathcal{V}\)是一个线性变换\((\dim\mathcal{U} = \dim \mathcal{V})\), 
        则\(\det f\)是一个数, 它的大小为\(f(\boldsymbol{e}_1) \wedge f(\boldsymbol{e}_2) \wedge \ldots \wedge f(\boldsymbol{e}_n)\)
        与\(\boldsymbol{\epsilon}_1 \wedge \boldsymbol{\epsilon}_2 \wedge \ldots \wedge \boldsymbol{\epsilon}_n\)的比值.
    \end{definition}

    我们抽象出矩阵的情况 (只有数).

    \begin{definition}
        [奇 (偶) 置换] 设\(\sigma\)是\(1,2,\ldots,n\)上的一个置换\((\)一一对应\()\).
        假设它能通过奇数次对换得到\(1,2,\ldots,n\), 则称\(\sigma\)为奇置换, 否则称为偶置换.\footnotemark{}
    \end{definition}
    \footnotetext{奇置换不能是偶置换的证明比较复杂, 可以使用逆序数证明, 不作赘述 (包括之前外积那段).}

    \begin{definition}
        [行列式] \(\mathbf{M} \in \mathbb{F}^{n \times n}\)
        \begin{equation}
            \det \mathbf{M} = \sum_{\sigma \in S_n} M_{\sigma(1)1} M_{\sigma(2)2} \cdots M_{\sigma(n)n}
        \end{equation}
        其中\(S_n\)表示全体置换, \(\sigma(i)\)表示置换后\(i\)的位置.
    \end{definition}

    这样, 行列式不为\(0\)的矩阵就是同构, 而行列式为\(0\)的矩阵就是会丢失维度的同态.

    另一方面, 考虑到外积事实上的结合律, 可以把行列式展开. (事实上考虑到分量是全部展开的\(\ldots \wedge \boldsymbol{v} \wedge \ldots\)
    关于\(\boldsymbol{v}\)构成一个线性映射, 我们可以先导出这个线性映射再计算行列式.)

    \begin{definition}
        [代数余子式] 设\(\mathbf{M} \in \mathbb{F}^{n \times n}\), 我们可以将矩阵
        去掉第\(i\)行第\(j\)列的元素, 得到一个\(n-1\)阶矩阵, 我们称这个矩阵的行列式为\(M_{ij}\)的余子式.
        将\(M_{ij}\)的余子式乘上\(- 1\)的\(i+j\)次幂, 我们得到的数叫做\(M_{ij}\)的代数余子式, 记作\(A_{ij}\).

        那么可以把\(\mathbf{M}\)单行展开, \(\det \mathbf{M} = \sum_{i=1}^{n} M_{ij}A_{ij} = \sum_{j=1}^{n} M_{ij}A_{ij}\)
    \end{definition}

    注意, 第一个式子是对列展开, 第二个式子是对行展开, 两者是等价的, 也显示出方阵行与列的对称性.

    同样的, 行列式也可以按多行展开

    \begin{definition}
        [多行展开] 设\(\mathbf{M} \in \mathbb{F}^{n \times n}\), 我们可以将矩阵去掉
        
        第\(i_{1},i_{2},\ldots,i_{k}\)行和\(j_{1},j_{2},\ldots,j_{k}\)列的元素 \((i_{1} < i_{2} < \ldots,j_{1} < j_{2} < \ldots)\), 得到一个\(n-k\)阶矩阵.
        乘上\(- 1\)的\(i_{1}+i_{2}+\cdots+i_{k}+j_{1}+j_{2}+\cdots+j_{k}\)次幂, 记作\(B\).
        挑出来的\(k\)行\(k\)列重复部分是个\(k \times k\)的矩阵, 乘上\(B\)记作\(C\).

        在选定\(k\)行后, 我们对全部可能的\(k\)列, 计算所得到的\(C\)值, 求和就是行列式的值.
    \end{definition}

    我们来考虑下述事实, 有\(\mathbf{M}\)和\(\mathbf{N}\)两个方阵, 每个方阵都对应了
    体积的变化, 那么如果复合两个变化, 体积倍数应该就是两个倍数的乘积.

    \begin{theorem}
        [行列式乘法] 设\(\mathbf{M} \in \mathbb{F}^{n \times n}\)和\(\mathbf{N} \in \mathbb{F}^{n \times n}\),
        那么\(\det(\mathbf{M} \mathbf{N}) = \det \mathbf{M} \det \mathbf{N}\).
    \end{theorem}

    行列式计算是有一定的技巧的, 可以对矩阵乘上一些已知行列式的矩阵, 常见的有:

    \begin{equation}
        \begin{bmatrix}
            1 & 0 & \cdots & 0 & \cdots & 0 \\
            0 & 1 & \cdots & 0 & \cdots & 0 \\
            \vdots & \vdots & \ddots & \vdots & \vdots \\
            0 & 0 & \cdots & \lambda & \cdots & 0 \\
            \vdots & \vdots & \vdots & \vdots & \ddots \\
            0 & 0 & \cdots & 0 & \cdots & 1
        \end{bmatrix}
    \end{equation}

    \begin{equation}
        \begin{bmatrix}
            1 & 0 & \cdots & 0 & \cdots & 0 & \cdots & 0 \\
            0 & 1 & \cdots & 0 & \cdots & 0 & \cdots & 0 \\
            \vdots & \vdots & \ddots & \vdots & \ddots & \vdots & \ddots & \vdots \\
            0 & 0 & \cdots & 1 & \cdots & a & \cdots & 0 \\
            \vdots & \vdots & \vdots & \vdots & \ddots & \vdots & \ddots & \vdots \\
            0 & 0 & \cdots & 0 & \cdots & 1 & \cdots & 0 \\
            \vdots & \vdots & \ddots & \vdots & \ddots & \vdots & \ddots & \vdots \\
            0 & 0 & \cdots & 0 & \cdots & 0 & \cdots & 1
        \end{bmatrix}
    \end{equation}

    \begin{equation}
        \begin{bmatrix}
            1 & 0 & \cdots & 0 & \cdots & 0 & \cdots & 0 \\
            0 & 1 & \cdots & 0 & \cdots & 0 & \cdots & 0 \\
            \vdots & \vdots & \ddots & \vdots & \ddots & \vdots & \ddots & \vdots \\
            0 & 0 & \cdots & 0 & \cdots & 1 & \cdots & 0 \\
            \vdots & \vdots & \vdots & \vdots & \ddots & \vdots & \ddots & \vdots \\
            0 & 0 & \cdots & 1 & \cdots & 0 & \cdots & 0 \\
            \vdots & \vdots & \ddots & \vdots & \ddots & \vdots & \ddots & \vdots \\
            0 & 0 & \cdots & 0 & \cdots & 0 & \cdots & 1
        \end{bmatrix}
    \end{equation}

    也就是说: 一行乘以\(\lambda\)倍, 一行乘以某个已知行列式的倍数, 两行交换, 对行列式的影响是确定的, 同理, 列变换也一样.

    有一行或列全为零, 那么行列式就是零. 对于行列式为\(0\)的矩阵, 如果把每一行都看作一个向量的表示, 那么所得向量组是线性相关的, 反之亦然.
    \begin{exercise}
        计算:
        \begin{equation}
            \det \begin{bmatrix}
                a_{11} & a_{12} \\
                a_{21} & a_{22}
            \end{bmatrix}
        \end{equation}
        \begin{equation}
            \det \begin{bmatrix}
                a_{11} & a_{12} & a_{13} \\
                a_{21} & a_{22} & a_{23} \\
                a_{31} & a_{32} & a_{33}
             \end{bmatrix}
        \end{equation}
    \end{exercise}

    \begin{exercise}
        计算: 
        \begin{equation}
            \det \begin{bmatrix}
                1 & x_1 & x_1^2 & \cdots & x_1^n \\
                1 & x_2 & x_2^2 & \cdots & x_2^n \\
                \vdots & \vdots & \vdots & \ddots & \vdots \\
                1 & x_n & x_n^2 & \cdots & x_n^n
            \end{bmatrix}
        \end{equation}

        特别的, 对于离散傅立叶变换, 有\(x_k = e^{\frac{2ik\pi}{n}}\).
    \end{exercise}


    \section{逆矩阵}

    现在假定行列式不为\(0\)的情况, 此时是同构映射. 既然如此, 我们可以考虑其逆映射与其对应的矩阵.

    \begin{definition}
        [逆矩阵] 设\(\mathbf{M} \in \mathbb{F}^{n \times n}\)是一个同构矩阵, 那么存在一个矩阵\(\mathbf{M}^{-1}\)使得\(\mathbf{M} \mathbf{M}^{-1} = \mathbf{M}^{-1} \mathbf{M} = \mathbf{I}\).
        我们称\(\mathbf{M}^{-1}\)为\(\mathbf{M}\)的逆矩阵, 其中\(\mathbf{I}\)是单位矩阵, 就是对角线为\(1\)其他地方为\(0\)的矩阵, 单位矩阵对应了一个单位映射.
    \end{definition}

    逆矩阵显然有如下性质.

    \begin{theorem}
        [逆矩阵性质] 设\(\mathbf{M} \in \mathbb{F}^{n \times n}\)是一个同构矩阵, 那么有如下性质:
        \begin{enumerate}
            \item \(\det \mathbf{M}^{-1} = \frac{1}{\det \mathbf{M}}\)
            \item \({(\mathbf{A}\mathbf{B})}^{-1} = \mathbf{B}^{-1} \mathbf{A}^{-1}\)
        \end{enumerate}
    \end{theorem}

    \begin{theorem}
        [逆矩阵唯一性] 假设\(\mathbf{B}, \mathbf{C}\)都是\(\mathbf{A}\)的逆矩阵.
        
        \begin{equation}
            (\mathbf{B} \mathbf{A}) \mathbf{C} = \mathbf{C} = \mathbf{B} (\mathbf{A} \mathbf{C}) = \mathbf{B}
        \end{equation}
    \end{theorem}
        
    接下来我们来考虑数值计算逆矩阵的问题.

    \begin{example}
        计算矩阵 \(\mathbf{M} = \begin{bmatrix}
            1 & 1 & 1 \\
            1 & 2 & 4 \\
            1 & 3 & 9
        \end{bmatrix}\)的逆矩阵.

        第一步, 先在右边加上单位矩阵, 得到
        \begin{equation}
            \begin{bmatrix}
                1 & 1 & 1 & 1 & 0 & 0 \\
                1 & 2 & 4 & 0 & 1 & 0 \\
                1 & 3 & 9 & 0 & 0 & 1
            \end{bmatrix}
        \end{equation}
        第二步, 进行行变换, 可行的行变换有:
        \begin{enumerate}
            \item 交换两行
            \item 某一行乘以一个非零常数
            \item 某一行加另一行的某个常数倍
        \end{enumerate}

        我们让第二三行减去第一行得到:
        \begin{equation}
            \begin{bmatrix}
                1 & 1 & 1 & 1 & 0 & 0 \\
                0 & 1 & 3 & -1 & 1 & 0 \\
                0 & 2 & 8 & -1 & 0 & 1
            \end{bmatrix}
        \end{equation}
        第三步, 让第三行减去第二行的两倍得到:
        \begin{equation}
            \begin{bmatrix}
                1 & 1 & 1 & 1 & 0 & 0 \\
                0 & 1 & 3 & -1 & 1 & 0 \\
                0 & 0 & 2 & 1 & -2 & 1
            \end{bmatrix}
        \end{equation}
        第四步, 让第二行减去第三行的二分之三倍得到:
        \begin{equation}
            \begin{bmatrix}
                1 & 1 & 1 & 1 & 0 & 0 \\
                0 & 1 & 0 & -\frac{5}{2} & 4 & -\frac{3}{2} \\
                0 & 0 & 2 & 1 & -2 & 1
            \end{bmatrix}
        \end{equation}
        第五步, 先让第三行除以二, 再让第一行减去第二三行得到:
        \begin{equation}
            \begin{bmatrix}
                1 & 0 & 0 & 3 & -3 & 1 \\
                0 & 1 & 0 & -\frac{5}{2} & 4 & -\frac{3}{2} \\
                0 & 0 & 1 & \frac{1}{2} & -1 & \frac{1}{2}
            \end{bmatrix}
        \end{equation}
    \end{example}

    上述过程是正确的, 为了方便叙述, 我们可以将矩阵分块
    
    \begin{definition}
        [分块矩阵] 当我们只考虑块的性质时, 我们可以将矩阵的一些行列看作是一块, 称作分块矩阵.
        \begin{equation}
            \mathbf{M} = \begin{bmatrix}
                \mathbf{A} & \mathbf{B} \\
                \mathbf{C} & \mathbf{D}
            \end{bmatrix} = \left [ \begin{tblr}{cccc|cccc}
                A_{11} & A_{12} & \cdots & A_{1m} & B_{11} & B_{12} & \cdots & B_{1n} \\
                A_{21} & A_{22} & \cdots & A_{2m} & B_{21} & B_{22} & \cdots & B_{2n} \\
                \vdots & \vdots & \ddots & \vdots & \vdots & \vdots & \ddots & \vdots \\
                A_{p1} & A_{p2} & \cdots & A_{pm} & B_{p1} & B_{p2} & \cdots & B_{pn} \\
                \hline
                C_{11} & C_{12} & \cdots & C_{1m} & D_{11} & D_{12} & \cdots & D_{1n} \\
                C_{21} & C_{22} & \cdots & C_{2m} & D_{21} & D_{22} & \cdots & D_{2n} \\
                \vdots & \vdots & \ddots & \vdots & \vdots & \vdots & \ddots & \vdots \\
                C_{q1} & C_{q2} & \cdots & C_{qm} & D_{q1} & D_{q2} & \cdots & D_{qn}
            \end{tblr} \right ]
        \end{equation}
        当然, 我们也可以分\(6\)块, 等等.
    \end{definition}

    我们来解释上述过程为什么成立:
    \begin{equation}
        \mathbf{T} \begin{bmatrix}
            \mathbf{A} & \mathbf{B}
        \end{bmatrix} = \begin{bmatrix}
            \mathbf{T} \mathbf{A} & \mathbf{T} \mathbf{B}
        \end{bmatrix}
    \end{equation}

    有不变量\(\mathbf{B}^{-1}\mathbf{A}\), 同样的, 也可以通过列的办法计算.

    \begin{exercise}
        计算二三阶方阵的逆矩阵, 尝试用行列式简化逆矩阵的表示.
    \end{exercise}

    \begin{theorem}
        [逆矩阵] \(\mathbf{M} \in \mathbb{F}^{n \times n}\), 则\({(M^{-1})}_{ij}\)
        为\(M_{ij}\)的余子式 (不妨记作\(P_{ij}\)) 除以\(\det \mathbf{M}\).
    \end{theorem}

    考考虑第\(i\)行的元素, 如果刚好是第\(i\)列, 乘法所得就是行列式的单行展开.
    如果不是, 则将余子式按\(j\)行展开, 易得其为零.

    从而, 行列式非零的矩阵有逆, 于是, 我们就可以证明:
    \begin{equation}
        \mathbf{B}\mathbf{A} = \mathbf{I} \leftrightarrow \mathbf{A}\mathbf{B} = \mathbf{I}
    \end{equation}

    \begin{equation}
        \mathbf{A} \mathbf{B} = \mathbf{B}^{-1} \mathbf{B} \mathbf{A} \mathbf{B} = \mathbf{B}^{-1} \mathbf{I} \mathbf{B} = \mathbf{I}
    \end{equation}

    \begin{exercise}
        计算傅立叶矩阵的逆矩阵.
    \end{exercise}

    \begin{exercise}
        给定一些未知数\(x_i\), 他们的全体线性函数\(f(x_i) = \sum \lambda_i x_i\)构成一个线性空间, 试推导多元一次方程组的解.
    \end{exercise}

    \begin{exercise}
        [马尔可夫链] 计算扔硬币连续扔出\(4\)次正面的期望扔硬币总次数, 你可以试着设出各种状态下的期望.
    \end{exercise}

    \begin{exercise}
        [斐波那契数列] 列出每年新兔子老兔子和去年新兔子老兔子的关系, 并用矩阵表达求根公式.
    \end{exercise}

    \section{矩阵的转置}

    我们来考虑矩阵变换带来的对偶空间的变换.

    假设\(\mathbf{M}\)是一个矩阵, 对应\(f:\mathcal{V} \to \mathcal{U}\), 我们试图
    寻找\(g:\mathcal{U}^* \to \mathcal{V}^*\), 满足\(f(\boldsymbol{A}) \boldsymbol{B} = g(\boldsymbol{B}) \boldsymbol{A}\).

    我们只考虑单位元素的问题, 即\(\boldsymbol{e}_i\)与\(\boldsymbol{\epsilon}^*_j\)的内积, 得到\(g\)所代表的矩阵\(\mathbf{N}\)满足
    \(N_{ij} = M_{ji}\), 此时, 我们称\(\mathbf{N}\)为\(\mathbf{M}\)的转置矩阵, 记作\(\mathbf{M}^T\).

    可以证明:

    \begin{equation}
        {(\mathbf{A} \mathbf{B})}^T = \mathbf{B}^T \mathbf{A}^T
    \end{equation}

    特别的, 在复数情况下, 对偶空间上的矩阵除了转置还多了一层共轭.

    \section{矩阵的秩}

    我们来考虑一些行列式为\(0\), 甚至不是方阵的矩阵.

    显然\(\mathbb{F}^n\)上有自然的线性空间结构, 采取这种情况, 那么一个矩阵可以分解成行向量与列向量的形式.

    \begin{definition}
        [矩阵的秩] 矩阵的秩是指其行向量组与列向量组所张成空间的维数, 用\(\rank\)表示.
    \end{definition}

    为了秩的存在, 我们需要证明两个维数是相同的.

    \begin{definition}
        [正交补] 对\(\mathcal{V}\)的一个子空间\(\mathcal{U}\), 可以定义它的正交补\(\mathcal{U}^\bot \subseteq \mathcal{V}^*\), 
        满足:
        \begin{equation}
            \boldsymbol{u} \in \mathcal{U}^\bot, \forall \boldsymbol{u} \in \mathcal{V}^*, \boldsymbol{u} (\boldsymbol{v}) = 0
        \end{equation}
    \end{definition}

    \begin{exercise}
        证明:
        \begin{equation}
            \dim \mathcal{U} + \dim \mathcal{U}^\bot = \dim \mathcal{V}
        \end{equation}
    \end{exercise}

    \begin{theorem}
        [秩的存在] 设\(\mathbf{M}\)是一个矩阵, 则其秩是存在的, 记其代表的线性变换为\(f\), 其转置代表的线性变换为\(f^*\).
        
        显然有: \(\ker f^* = {(\Im f)}^\bot\).

        利用此式与上面练习的式子可以证明\(\dim \Im f = \dim \Im f^*\).
    \end{theorem}

    行列的对称性在此处又体现出来了.

    \begin{exercise}
        证明: \(\mathbf{M}\)乘上一个可逆矩阵秩不变.
    \end{exercise}

\end{document}
